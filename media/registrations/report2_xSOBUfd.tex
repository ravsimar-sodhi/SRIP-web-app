\documentclass[a4paper]{article}

\usepackage[english]{babel}
\usepackage[utf8]{inputenc}
\usepackage{hyperref}
% \usepackage{geometry}

\begin{document}
\begin{titlepage} % Suppresses displaying the page number on the title page and the subsequent page counts as page 1
	\newcommand{\HRule}{\rule{\linewidth}{0.5mm}} % Defines a new command for horizontal lines, change thickness here

	\center % Centre everything on the page
	\textsc{\LARGE Honours Project - II}\\[1.5cm] % Main heading such as the name of your university/college

	\vfill
	% \textsc{\Large Assignment No.}\\[0.5cm] % Major heading such as course name

	% \textsc{\large Peer reviewed paper summary}\\[0.5cm] % Minor heading such as course title

	\HRule\\[0.4cm]

	{\huge\bfseries Final Report}\\[0.4cm] % Title of your document
	\HRule\\[1.5cm]

	\begin{minipage}{0.4\textwidth}
		\begin{flushleft}
			\large
			\textit{By}\\
			\textsc{Ravsimar Singh} % Your name
		\end{flushleft}
	\end{minipage}
	~
	\begin{minipage}{0.4\textwidth}
		\begin{flushright}
			\large
			\textit{Advisor}\\
			\textsc{Prof. Radhika Mamidi} % Supervisor's name
		\end{flushright}
	\end{minipage}

	\vfill\vfill\vfill % Position the date 3/4 down the remaining page

	% {\large\today} % Date, change the \today to a set date if you want to be precise

	\vfill % Push the date up 1/4 of the remaining page

\end{titlepage}

\section{Overview}\label{sec:overview}
For this semester, we've done two major tasks. First, we implemented the paper \cite{DBLP:journals/corr/VakulenkoS17} and tried to use it on our data. The results were not very good, so we moved on to a rule based approach.
Second, our rule based approach. We are working on IPL-2018 data, which we have scraped ourselves from ESPN. The data exists in tabular form.

This report describes overviews of research papers that were read, and description of the system we are worked on.

All papers are properly referenced at the bottom.

\section{Dataset}\label{sec:dataset}
We obtained the data from ESPN\footnote{http://www.espncricinfo.com} using scraping. We obtained individual scorecards of each match, which contained the score,runs,fours,sixes, strike rate etc. of each batsman, and the corresponding info for each bowler. We also obtained general data for the whole season, information containing when and where each match was held, and between which teams, and who was the winner of the match.

\section{TableQA}\label{sec:tableQA}
The End-To-End Memory network architecture \cite{NIPS2015_5846} is employed to transform the natural-language questions into the table lookups. Memory Network \cite{DBLP:journals/corr/WestonCB14} is a recurrent neural network (RNN) trained to predict the correct answer by combining continuous representations of an input table and a question. It consists of a sequence of memory layers that allow to go over the content of the input table several times and perform reasoning in multiple steps.

The input tables, questions, and answers are embedded into a vector space using a bag-of-words models, which neglects the ordering of words.
The output layer generates the predicted answer to the input question and is implemented as a softmax function in the size of the vocabulary, i.e. it outputs the probability distribution over all possible answers, which could be any of the table cells.

The network is trained using stochastic gradient descent with linear start to avoid the local minima. The objective function is to minimize the cross-entropy loss between the predicted answer and the true answer from the training set.

\subsection{Synthetic Data}\label{subsec:synthdata}
As implemented in the paper itself \cite{DBLP:journals/corr/VakulenkoS17}, since we had very less data in the form of tables, we generate synthetic data.
The template of question formed were:
\begin{itemize}
	\item What is the \{\} by \{\} ?
\end{itemize}
where \{\} was replaced by a suitable column of the table. An example question would be, "What is the s1core by MS Dhoni?".

\subsection{Results}\label{subsec: results}
The score we achieved on our test data was around 33.2\%. This low score could have been caused by the synthetic data used, and maybe score could have been improved if more data had been available to us.

Next, we moved on to a rule-based approach.
\section{Rule-based Approach}\label{sec:rbapproach}
Although there exist a lot of rule-based QA system, very few exist on tabular data. Most approaches which work on tabular data are in the form of NLIDBs.

Our procedure is as follows:
\begin{enumerate}
	\item Query Processing
	\item Table Lookup
\end{enumerate}

Some of the papers we referred to were \cite{quarc},\cite{DWIVEDI2013417}.
We also made a list of relevant questions which should be addressed in cricket-based QA system. The list can be found \href{https://docs.google.com/document/d/1Iiwfx7Hm9N0VpjTHIgSzxjeoOtr6RSk1FLHRPZb53Xo/edit?usp=sharing}{here}.
All our code can be found \href{https://github.com/ravsimar-sodhi/cricket-scoreboard-scrape}{here}

\subsection{Query Processing}\label{subsec:queryproc}
Our input is a natural language question. First, we tokenize the question and keep all the words. However, tokenizing is itself not sufficient, since some words need to be chunked. For example, a team name, say "Chennai Super Kings", needs to be chunked as one token.
We obtain POS tags of the initial tokens, and using a Parser and grammar rules,
we obtain a parse tree with the chunks we need.

Dates in the question also needs to be handled as one chunk, since questions of the form such as "Who won between Chennai Super Kings \& Delhi Daredevils on April 14, 2018?".

Currently, we have written rules for some templates of questions:
\begin{itemize}
	\item Who was the \emph{winner/loser} between \emph{Team 1} and \emph{Team 2} on \emph{Date}?
	\item Who was the \emph{winner/loser} between \emph{Team 1} and \emph{Team 2} at \emph{Ground}?
	\item Where was the match held between \emph{Team 1} and \emph{Team 2} on \emph{Date}?
	\item When was the match held between \emph{Team 1} and \emph{Team 2} at \emph{Ground}?
	\item What were the number of \emph{runs/fours/sixes/balls played} by \emph{Player} in the match between \emph{Team 1} and \emph{Team 2} on \emph{Date}?
	\item What were the number of \emph{runs/fours/sixes/balls played} by \emph{Player} in the match between \emph{Team 1} and \emph{Team 2} at \emph{Ground}?
	\item What were the number of \emph{wickets taken/balls bowled/overs bowled} by \emph{Player}?
\end{itemize}

\bibliography{bibl}
\bibliographystyle{apalike}
\end{document}
